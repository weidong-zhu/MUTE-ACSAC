% \documentclass[conference,compsoc]{IEEEtran}

% \PassOptionsToPackage{hyphens}{url}

\documentclass[conference,compsoc]{IEEEtran}

\PassOptionsToPackage{hyphens}{url}
%\renewcommand\footnotetextcopyrightpermission[1]{}
\usepackage{epsfig,authblk}
\usepackage{etex}
\pagestyle{plain}
%%
%% \BibTeX command to typeset BibTeX logo in the docs
% \AtBeginDocument{%
%   \providecommand\BibTeX{{%
%     Bib\TeX}}}



% % % % % % % % % % % % % % %
% PACKAGE USAGES
% % % % % % % % % % % % % % %
\usepackage{algorithm,algorithmicx,algpseudocode}
%\usepackage{afterpage}
\usepackage{amstext}   % provides \text{} command in math mode
\usepackage{amsmath}
%\usepackage{amssymb}% http://ctan.org/pkg/amssymb
%\usepackage{authblk}
\usepackage{lipsum}
\usepackage{amsthm}
\usepackage{float} % custom listing float and 'H' here
\usepackage{balance}
%\usepackage{caption}
\usepackage{color}
\usepackage{booktabs}
\usepackage{comment}

\let\labelindent\relax% https://tex.stackexchange.com/questions/170772/command-labelindent-already-defined
\usepackage{enumitem}

\usepackage{epic}
\usepackage{epsf}
\usepackage{epsfig}
\usepackage[T1]{fontenc}% http://ctan.org/pkg/fontenc
\usepackage{graphicx}% http://ctan.org/pkg/graphicx
%\usepackage{hyperref}
\usepackage{latexsym}
%\usepackage[colorlinks=true,citecolor=black,filecolor=black,linkcolor=black,urlcolor=black]{hyperref}
\usepackage{hyperref}
\hypersetup{                    % ...like so
	colorlinks=true,
	linkcolor={blue!70!black},
	citecolor=black,
	urlcolor={blue!70!black}
}
\usepackage{listings}
\usepackage{makecell}% http://ctan.org/pkg/makecell
\usepackage{multirow}
\usepackage{multicol}
\usepackage{pifont}% http://ctan.org/pkg/pifont
\usepackage{soul}
% \usepackage[caption=false,font=normalsize,labelfont=sf,textfont=sf]{subfig}
\usepackage[subrefformat=parens,font=normalsize,labelformat=parens]{subfig}
\usepackage{textcomp}
\usepackage{tabularx}
\usepackage{xurl}
\usepackage{xspace}    % sticks a sane space after a command
\usepackage{tikz}
\usepackage{pgfplots}
\usepackage[color=white]{todonotes}
\usepackage[utf8]{inputenc} % for hevea
\usepackage{algpseudocode}
\usetikzlibrary{patterns}



% \titlespacing*{\section}
% {0pt}{2.5ex}{1.2ex}
% \titlespacing*{\subsection}
% {0pt}{1.5ex}{0.7ex}
\usepackage{caption}
% \captionsetup[table]{skip=2pt}
% \addtolength\abovecaptionskip{-7pt}
%\addtolength\belowcaptionskip{-8pt}
% \captionsetup[subfloat]{font=scriptsize,captionskip=-1pt}
%\usepackage[all=normal,floats=tight,mathspacing=tight,wordspacing=tight]{savetrees}
%\usepackage{cite}
%\hyphenation{op-tical net-works semi-conduc-tor IEEE-Xplore}
%\def\BibTeX{{\rm B\kern-.05em{\sc i\kern-.025em b}\kern-.08em
%		T\kern-.1667em\lower.7ex\hbox{E}\kern-.125emX}}
%	\usepackage{balance}
% % % % % % % % % % % % % % %
% PACKAGE CUSTOMIZATION
% % % % % % % % % % % % % % %
% \def\BibTeX{{\rm B\kern-.05em{\sc i\kern-.025em b}\kern-.08em
%     T\kern-.1667em\lower.7ex\hbox{E}\kern-.125emX}}
%\lstset{
%  %language=python,
%  numbers=none,
%  %style=Prolog-pygsty,
%  basicstyle={\linespread{0.6}\tt\small},
%  stringstyle=\color{strings},
%  commentstyle=\color{comments},
%  emph      = [1]{
%    allow, type_transition,
%    usb,
%    bt, hci, l2cap,
%    nfc},
%  emphstyle = [1]{\ttfamily\bfseries},
%  stepnumber=1,                   % the step between two line-numbers.
%  numbersep=5pt,                  % how far the line-numbers are from the code
%  backgroundcolor=\color{white},  % choose the background color. You must add \usepackage{color}
%  showspaces=false,               % show spaces adding particular underscores
%  showstringspaces=false,         % underline spaces within strings
%  showtabs=false,                 % show tabs within strings adding particular underscores
%  tabsize=2,                      % sets default tabsize to 2 spaces
%  captionpos=b,                   % sets the caption-position to bottom
%  breaklines=true,                % sets automatic line breaking
%  breakatwhitespace=true,         % sets if automatic breaks should only happen at whitespace
%  xleftmargin=1em,framexleftmargin=1.5em
%}%

%\newfloat{lstfloat}{htbp}{lop}
%\floatname{lstfloat}{Listing}


\renewcommand{\sectionautorefname}{Section}
\newcommand{\algorithmautorefname}{Algorithm}
\renewcommand{\subsectionautorefname}{Section}
\newcommand{\lstfloatautorefname}{Listing}
\newcommand{\ignoreme}[1]{}

\newcommand{\subfigureautorefname}{\figureautorefname}

\newtheorem{theorem}{Theorem}
\newtheorem{lemma}{Lemma}
\newtheorem{definition}{Definition}

% % % % % % % % % % % % % % %
% CUSTOM COMMANDS
% % % % % % % % % % % % % % %
\definecolor{cb1}{RGB}{0, 73, 73}      % Dark teal
\definecolor{cb7}{RGB}{255, 109, 182}  % Pink
\definecolor{cb3}{RGB}{255, 128, 14}   % Orange
\definecolor{cb4}{RGB}{171, 171, 0}    % Olive
\definecolor{cb5}{RGB}{89, 89, 89}     % Dark gray
\definecolor{cb6}{RGB}{217, 95, 2}     % Reddish orange
\definecolor{cb9}{RGB}{0, 128, 43}     % Green
\definecolor{cb8}{RGB}{105, 179, 216}  % Sky blue
\definecolor{cb2}{RGB}{168, 120, 110}  % Brownish

\newcommand*\circled[1]{\tikz[baseline=(char.base)]{
		\node[shape=circle,draw,inner sep=0.1pt] (char) {#1};}}
	
\newcommand*\bcircled[1]{\tikz[baseline=(char.base)]{
		\node[shape=circle,fill,inner sep=0.1pt] (char) {\textcolor{white}{#1}};}}

\newcommand*\gcircled[1]{\tikz[baseline=(char.base)]{
		\node[shape=circle,draw,fill={rgb:black,1;white,2},inner sep=0.1pt] (char) {{#1}};}}

%\usepackage[disable]{todonotes}
\newcommand{\assignment}[2]{\todo[inline,author=\color{blue}{\bfseries Assignment \underline{#1}}]{\color{black}#2}}
%\newcommand{\outline}[1]{(\emph{{\bf Outline}: #1})}
%\newcommand{\outline}[1]{}

% \renewcommand{\rothead}[2][60]{\makebox[9mm][c]{\rotatebox{#1}{\makecell[c]{#2}}}}%

% \newcommand{\cmark}{\ding{51}}%
% \newcommand{\xmark}{\ding{55}}%
% \renewcommand{\rothead}[2][60]{\makebox[9mm][c]{\rotatebox{#1}{\makecell[c]{#2}}}}%

%\newcommand{\grant}[1]{{\color{blue}{\em[TODO($Grant$) - #1]}}}
% \newcommand{\tocite}[1]{{\color{purple}{[\textbf{Cite:} #1]}}}

%\renewcommand{\algorithmicrequire}{\textbf{Input:}}

% Disable TODOs
%\renewcommand{\grant}[1]{}

%\newcommand{\rephrase}[1]{ [This passage is self-plagiarized needs to be reworked: {\color{red}#1}]}
\newcommand{\ws}[1]{{\color{white}#1}}

% Place your quick name macros here


\newcommand{\weidong}[1]{{\color{blue}{\em[$Weidong$ - #1]}}}
\newcommand{\wbnote}[1]{\ifx\outforreview\undefined\refstepcounter{mynote}{\it\textcolor{green}{(WB~\thenote: { #1})}}\fi}

\usepackage{mathtools}
\DeclarePairedDelimiter\ceil{\lceil}{\rceil}
\DeclarePairedDelimiter\floor{\lfloor}{\rfloor}

% % % % % % % % % % % % % % %
% DOCUMENT HACKS
% % % % % % % % % % % % % % %

%\makeatletter
%\def\blfootnote{\xdef\@thefnmark{}\@footnotetext}
%\makeatother
%
%\makeatletter
%\newcommand\footnoteref[1]{\protected@xdef\@thefnmark{\ref{#1}}\@footnotemark}
%\makeatother
%
%\makeatletter
%\newcommand{\labitem}[2]{%
%\def\@itemlabel{\textbf{#1}}
%\item
%\def\@currentlabel{#1}\label{#2}}
%\makeatother

% % % % % % % % % % % % % % %
% TITLE AND AUTHORS
% % % % % % % % % % % % % % %

%\title{NASA: NVM-Assisted Secure Deletion for Flash Memory }
%%\date{}
%\author{Anonymous}

%\author[1]{Weidong Zhu}
%\author[1]{Grant Hernandez}
%\author[1]{Kevin R. B. Butler}
%\affil[1]{\normalsize Florida Institute for Cyber Security (FICS) Research, University of
%  Florida, Gainesville, FL, USA \authorcr
%\textit{\small \{grant.hernandez,butler\}@ufl.edu}\vspace{1mm}}
%\affil[2]{\normalsize Institution 2 \authorcr
%  \textit{\small email@example.com}\vspace{1mm}}

% % % % % % % % % % % % % % %
% DOCUMENT BODY
% % % % % % % % % % % % % % %

%\sloppy
%
%\AtBeginDocument{%
%	\providecommand\BibTeX{{%
%			\normalfont B\kern-0.5em{\scshape i\kern-0.25em b}\kern-0.8em\TeX}}}

%% Rights management information.  This information is sent to you
%% when you complete the rights form.  These commands have SAMPLE
%% values in them; it is your responsibility as an author to replace
%% the commands and values with those provided to you when you
%% complete the rights form.
%\setcopyright{acmcopyright}
%\copyrightyear{2021}
%\acmYear{2021}
%\acmDOI{10.1145/1122445.1122456}

%% These commands are for a PROCEEDINGS abstract or paper.
%\acmConference[EUROSYS '22]{EUROSYS '22}{April 05--08, 2022}{Rennes, France
%}
%\acmBooktitle{EUROSYS '22,
%	April 05--08, 2022, Rennes, France
%}
%\acmPrice{}
%\acmISBN{}

%\hyphenation{op-tical net-works semi-conduc-tor}

\begin{document}


\begin{figure*}
	
	\minipage[t]{0.32\textwidth}
	%\vspace{0pt}
		\begin{tikzpicture}
	\begin{axis} [
	ybar,
	style={font=\scriptsize},
	symbolic x coords={SW,SR,RW,RR},
%	skip coords between index={5}{9},
	%		nodes near coords align={vertical},
	y label style={at={(axis description cs:0.18,.5)},anchor=south},
	ylabel style={align=center},
	ylabel={Throughput (MB/s)},
	ymajorgrids,
	ytick style={draw=none},
	ymin=0,
	ymax=320,
	ytick={0,160,320},
	xtick=data,
	ybar=0.05pt,
	height=2.8cm, width=6.2cm,
	legend style={fill=none,draw=none,at={(1.8,1.4)},anchor=north,legend columns=8,nodes={scale=0.9, transform shape}},
	legend cell align={left},
	legend image code/.code={
		\draw [#1] (0cm,-0.1cm) rectangle (0.1cm,0.1cm); }
	legend cell align={left},
	bar width=0.12cm,
	xticklabel style={outer sep=-3pt},
	xtick style={draw=none},
	xtick distance={120},
	enlarge x limits=0.18,
	]
	
	%\addplot [fill=cb1,postaction={pattern=horizontal lines}] table[x=trace,y=SSD-NoFDE,col sep=comma] {results_csv/bw.csv};
        \addplot [fill=cb2,postaction={pattern=grid}] table[x=trace,y=SSD-FDE,col sep=comma] {results_csv/bw.csv};
        \addplot [fill=cb1,postaction={pattern=north east lines}] table[x=trace,y=PEARL-P,col sep=comma] {results_csv/bw.csv};
	\addplot [fill=cb4,postaction={pattern=crosshatch}] table[x=trace,y=PEARL-H,col sep=comma] {results_csv/bw.csv};
 	\addplot [fill=cb5,postaction={pattern=dots}] table[x=trace,y=MDEFTL-P,col sep=comma] {results_csv/bw.csv};
	\addplot [fill=cb6,postaction={pattern=bricks}] table[x=trace,y=MDEFTL-H,col sep=comma] {results_csv/bw.csv};
	\addplot [fill=cb7,postaction={pattern=north west lines}] table[x=trace,y=MUTE-PO,col sep=comma] {results_csv/bw.csv};
	\addplot [fill=cb8,postaction={pattern=horizontal lines}] table[x=trace,y=MUTE-PH,col sep=comma] {results_csv/bw.csv};
	\addplot [fill=cb9,postaction={pattern=}] table[x=trace,y=MUTE-H,col sep=comma] {results_csv/bw.csv};
	%,nodes near coords,every node near coord/.append style={font=\scriptsize}
	\legend{Baseline,PEARL-P,PEARL-H,MDEFTL-P,MDEFTL-H,MUTE-PO,MUTE-PH,MUTE-H}
	\end{axis}
	\end{tikzpicture}

	\vspace{-1.5em}	
	\caption{{The bandwidth of Baseline, PEARL, MDEFTL and MUTE when testing FIO benchmarks.}}
	%	\vspace{-0.7em}
	\label{bandwidth}
	%		\vspace{-1em}
	\endminipage\hfill
	\minipage[t]{0.32\textwidth}
	%\vspace{0pt}
		\begin{tikzpicture}
	\begin{axis} [
	ybar,
	style={font=\scriptsize},
	symbolic x coords={hm,prxy,rsrch,wdev},
%	skip coords between index={5}{9},
	%		nodes near coords align={vertical},
	y label style={at={(axis description cs:0.23,.5)},anchor=south},
	ylabel style={align=center},
	ylabel={Latency (ms)},
	ymajorgrids,
	ytick style={draw=none},
	ymin=0,
	ymax=6,
	ytick={0,3,6},
	xtick=data,
	ybar=0.05pt,
	height=2.8cm, width=6.2cm,
	legend style={fill=none,draw=none,at={(0.5,1.4)},anchor=north,legend columns=4,nodes={scale=0.8, transform shape}},
	legend cell align={left},
	legend image code/.code={
		\draw [#1] (0cm,-0.1cm) rectangle (0.1cm,0.1cm); }
	legend cell align={left},
	bar width=0.12cm,
	xticklabel style={outer sep=-9pt,text height=2.2ex},
	xtick style={draw=none},
	xtick distance={120},
	enlarge x limits=0.18,
	]

	%\addplot [fill=cb1,postaction={pattern=horizontal lines}] table[x=trace,y=SSD-NoFDE,col sep=comma] {results_csv/latency.csv};
        \addplot [fill=cb2,postaction={pattern=grid}] table[x=trace,y=SSD-FDE,col sep=comma] {results_csv/latency.csv};
        \addplot [fill=cb1,postaction={pattern=north east lines}] table[x=trace,y=PEARL-P,col sep=comma] {results_csv/latency.csv};
	\addplot [fill=cb4,postaction={pattern=crosshatch}] table[x=trace,y=PEARL-H,col sep=comma] {results_csv/latency.csv};
 	\addplot [fill=cb5,postaction={pattern=dots}] table[x=trace,y=MDEFTL-P,col sep=comma] {results_csv/latency.csv};
	\addplot [fill=cb6,postaction={pattern=bricks}] table[x=trace,y=MDEFTL-H,col sep=comma] {results_csv/latency.csv};
	\addplot [fill=cb7,postaction={pattern=north west lines}] table[x=trace,y=MUTE-PO,col sep=comma] {results_csv/latency.csv};
	\addplot [fill=cb8,postaction={pattern=horizontal lines}] table[x=trace,y=MUTE-PH,col sep=comma] {results_csv/latency.csv};
        \addplot [fill=cb9,postaction={pattern=}] table[x=trace,y=MUTE-H,col sep=comma] {results_csv/latency.csv};
	% \addplot [fill=cb9,postaction={pattern=},nodes near coords,nodes near coords align={vertical},every node near coord/.append style={font=\scriptsize}] table[x=trace,y=MUTE-H,col sep=comma] {results_csv/latency.csv};

	
	%\legend{Baseline,MUTE-PO,MUTE-PH,MUTE-H}
	\end{axis}
	\end{tikzpicture}

	\vspace{-0.3em}	
	\caption{{The average latency of Baseline, PEARL, MDEFTL and MUTE when evaluating MSR workloads.}}
	%	\vspace{-0.7em}
	\label{response_time}
	%		\vspace{-1em}
	\endminipage\hfill
	\minipage[t]{0.32\textwidth}
	%\vspace{0pt}
	\input{tikz_scripts/waf.tikz}
	\vspace{-0.3em}	
	\caption{{The average WAF of Baseline, PEARL, MDEFTL and MUTE when evaluating MSR workloads.}}
	
	\label{waf-msr}
	\vspace{-2em}
	\endminipage\hfill
	
	% 	\minipage[t]{0.23\textwidth}
	%         %\vspace{0pt}
	% 	\input{tikz_scripts/lat_portion.tikz}
	% 	\vspace{-1.7em}	
	% 	\caption{{The latency percentage of MUTE-H in different operations.}}
	% %	\vspace{-0.7em}
	% 	\label{response_time}
	% 	%		\vspace{-1em}
	% 	\endminipage
	
	%\vspace{-1em}
\end{figure*}

\begin{figure}
	\centering
	\vspace{-0.5em}
	

\begin{tikzpicture}
\begin{axis}[
    style={font=\scriptsize},
    xlabel={Page size $P$ (KB)},
    ylabel={Capacity $\mathcal{V}_h$ (GB)},
    xlabel style={at={(axis description cs:0.5,-0.1)},anchor=south},
    ylabel style={at={(axis description cs:0.1,.5)},anchor=south},
    ylabel style={align=center},
    xtick={0,4,16,32,64},
    ytick={0,10,20,30,40,50},
    xtick pos=bottom,
    ytick pos=left,
    ymin=0,
    ymax=50, % Adjust based on your data
    height=3.3cm,
    width=9cm,
    legend style={fill=none,draw=none,at={(0.5,1.22)},anchor=north,legend columns=2,nodes={scale=1.0, transform shape}},
    legend cell align={left},
    legend image post style={xscale=0.9},
    cycle list name=color list,
    grid=major,
    thick,
    enlarge x limits=0.05,
    scatter/classes={%
        a={mark=square*,blue},%
        b={mark=triangle*,red}},
]

\addplot+[thick,cb1] table[x=P_KB,y=Y_b1,col sep=comma] {results_csv/new_cap_data.csv};
\addplot+[thick,cb2] table[x=P_KB,y=Y_b2,col sep=comma] {results_csv/new_cap_data.csv};

\legend{b = 128 bits, b = 256 bits}
\end{axis}
\end{tikzpicture}




	\vspace{-0.5em}
	\caption{Hidden volume capacity $\mathcal{V}_h$ for different page size $P$ when total capacity $C$ is 512GB and $M$ is 56 bits}
	\vspace{-1em}
	\label{fig:capacity}
\end{figure}






\end{document}
